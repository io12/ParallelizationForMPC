\section{Analytical Model}
\label{sec:model}

%\ana{We need an intro to section here.}
%\ana{Terminology: "operation" or "instruction" or "gate"?}

This section presents a model to reason about the cost of execution of MPC programs, including accounting for amortization.
We define the assumptions and setting in~\secref{sec:mpc}. We proceed to define the scheduling problem in~\secref{sec:problem}, 
which we expected to be able to solve optimally. \secref{sec:np} shows that the problem is NP-hard via a reduction to the 
Shortest Common Supersequence (SCS) problem. Despite the negative general result, we expect the formulation in terms of SCS to 
be useful as sequences are short and few in practice.



\subsection{Scheduling in MPC}
\label{sec:mpc}

%\ana{Here we need more on MPC before jumping to scheduling. MPC-source, basic assumptions about static loop bounds, etc.}

For this treatment we make the following simplifying assumptions:

\begin{enumerate}
\item All statements in the program execute using the same protocol (sharing). That is, there is no share conversion.
%\ishaq{This is not an assumption, this is our setting. We work in the single protocol world.}
\item There are two tiers of MPC instructions, local and remote. A local instruction (e.g., ADD in Arithmetic, XOR in Boolean) 
has cost $\beta$ and a remote instruction (e.g., MUX, MUL, SHL, etc.) has cost $\alpha$, where $\alpha >> \beta$. We assume that all remote
instructions have the same cost.
%\ishaq{Don't we need to distinguish between local (cheap) vs. interactive (expensive) instructions? I don't see how we could assume this. -- Past suggestion from Vassilis: Ideally we should assume cost to be a convex function.}\ana{After discussion 1/12: replace single costs 1 with two levels of costs: $\alpha$ for non-local operations, e.g., MUL, and $\beta$ for local ops, e.g., ADD.}
\item  In MPC frameworks, executing $n$ operations ``at once'' in a single SIMD operation costs a lot less than executing those $n$ operations one by one.
Following Amdahl's law, we write $\alpha = \frac{1}{s}p\alpha + (1-p)\alpha$, where $p$ is the fraction of execution time that benefits from amortization and $(1-p)$
is the fraction that does not, and $s$ is the available resource. Thus, $n\alpha = \frac{n}{s}p\alpha + n(1-p)\alpha$.
For the purpose of the model we assume that $s$ is large enough and the term $\frac{n}{s}p\alpha$ amounts to a \emph{fixed cost} incurred regardless of
whether $n$ is $10,000$ or just $1$. (This models the cost of preparing and sending a packet from party A to party B for example.) Therefore, amortized execution
of $n$ operations is $f(n) = \alpha_\mathit{fix} + n\alpha_{var}$ in contrast to unamortized execution $g(n) = n\alpha_\mathit{fix} + n\alpha_{var}$.
We have $\alpha_\mathit{fix} << n\alpha_\mathit{fix}$ and since fixed cost dominates variable cost (particularly for remote operations), we have $f(n) << g(n)$.
%We assume infinite parallel capacity---i.e., a single MPC-instruction costs as much as $N$ amortized instructions, namely $\alpha$ or $\beta$.
%This is a standard assumption in Cryptographic Parallel RAM. ABY presents empirical support for this assumption~\ana{Add citations. PRAM, ABY.}
%\ishaq{Comment from Vassilis: We assume there are infinite parallel capacities, this is the standard assumption in PRAM (Cryptographic Parallel RAM).}\ana{PRAM assumption stays. We have to replace the unit cost of 1 with a suitable amortization function $f(n)$, which I don't think changes much in the cost modeling and analysis.}
\item MPC instructions scheduled in parallel benefit from amortization \emph{only if} they are the same instruction. Given our previous assumption,
2 MUL instructions can be amortized in a single SIMD instruction that costs $\alpha_\mathit{fix} + 2\alpha_\mathit{var}$, however a MUL and a MUX instruction
still cost $2\alpha_\mathit{fix} + 2\alpha_\mathit{var}$ even when scheduled ``in parallel''.\footnote{This is not strictly true, but assuming it, e.g. as in 
\cite{Ishaq:2019, NDSS:DemSchZoh15, Mohassel:2018}, helps simplify the problem.}
%MPC instructions scheduled in parallel benefit from amortization \emph{only if} they are the same instruction. Given our previous assumption,
%2 MUL instructions scheduled in parallel benefit from amortization and cost $\alpha$, however a MUL and a MUX instructions scheduled
%in parallel still cost $2\alpha$.
%\ishaq{We need to reword this assumption to something like "$K$ parallel MUL costs much less than $K$ (because they get amortized), but any mix of $K$  MUL and MUX still cost roughly $K$. Specifically, we should take away the low constants (2 and 1) because we know for low constants this is not true.} \ana{Again, core assumption that MUL and MUX don't benefit from amortization stays. We have to replace constant costs with functions, as in (3).}
\end{enumerate}

\subsection{Problem Statement}
\label{sec:problem}

\ana{Ishaq? Basically, define sequential schedule, then define an equivalent parallel schedule. A parallel schedule is equivalent if it preserves def-use relations in sequential schedule, or in other words, schedules def ahead of the use. Problem is to minimize cost of Parallel schedule.}

\ishaq{TODO: make it consistent with the next section.}

At the lowest level, we have two types of MPC instructions 1) local/non-interactive instruction (i.e. \add) and 2) remote/interactive instruction (i.e. \mul). Let the cost of a single \add instruction be given by a monotonically decreasing $f(n)$, where the argument $n$ is the number of \add instructions being executed in parallel. Similarly the cost of a single \mul is given by a monotonically decreasing function $g(n)$.

Given a serial schedule (a linear graph) of an MPC program i.e. a sequence of instructions $\mathcal{S}:=(S_1, \dots, S_n)$, where $S_i \in \{\text{\add, \mul}\}, 1 \leq i \leq n$, and a def-use dependency graph $G(V, E)$ corresponding to $\mathcal{S}$, our task is to construct a parallel schedule (another linear graph) $\mathcal{P} := (P_1, \dots, P_n)$ observing the following conditions:

\begin{enumerate}
    \item Multiple, not necessarily continuous instructions of the same type (i.e. either \add or \mul) from $\mathcal{S}$ can be grouped into a single $P_i$  However, all such instructions must be of the same type (either \add or \mul).
    \item Def-use dependencies of the graph $G(V, E)$ are should be preserved i.e. if instructions $S_i, S_j, i < j$ are a def-use (an edge exists from $S_i$ to $S_j$ in $G$), then they can only be mapped to $P_{i'}, P_{j'}, i' < j'$.
\end{enumerate}

Our goal is to construct minimize the height of the graph $\mathcal{P}$. Indeed, a graph $\mathcal{P}$ with minimum height will maximize parallelization.

\paragraph{Correctness} Correctness of $\mathcal{P}$ is guaranteed by definition. Since def-use dependencies are preserved, the function (being computed) remains the same.

\paragraph{Cost Comparison} For the sequential schedule $\mathcal{S}$ consisting of $L$ local and $R$ remote instructions, the total cost is $\mathit{cost}(\mathcal{S}) = L \cdot f(1) + R \cdot g(1)$. In the extreme case where all $L$ and all $R$ instructions can be parallelized, the cost of $\mathcal{P}$ is $\mathit{cost}(\mathcal{P}) = L \cdot f(L) + R \cdot g(R)$. Since both $f$ and $g$ are monotonically decreasing, $\mathit{cost}(\mathcal{P}) < \mathit{cost}(\mathcal{S})$. Cost of all other parallel schedules lies between the extremes of $\mathit{cost}(\mathcal{S})$ and $\mathit{cost}(\mathcal{P})$.

Note that we use an MPC-Source control flow graph (CFG) $G'(V', E')$ along with def-use graph $G(V,E)$ to construct $\mathcal{P}$. We consider a linearized MPC schedule $\mathcal{S}$ above for ease of exposition. The argument becomes slightly more involved when dealing with a graph $G'$ that may contain cycles.

% Given a def-use dependency graph $G(V, E)$ that is constructed from for the input program. Each vertex/node in this graph represents an operation/instruction in the program and an edge represents dependence i.e. an edge from $V_1$ to $V_2$ means the result of operation in $V_1$ is used in the operation in $V_2$. To build a parallel schedule, we translate this graph $G(V, E)$ into another dependency graph $G'(V', E')$ such that:

% \begin{enumerate}
%     \item We collapse multiple vertices of the same type (i.e. same operation) in $V$ into a single vertex in $V'$.
%     \item dependencies are not broken i.e. If there is an edge from $V_1$ to $V_2$ ($V_1, V_2 \in V$), and we put them in vertices $V'_1, V'_2 \in G'$ respectively, then an edge must exist from $V'_1$ to $V'_2$.
% \end{enumerate}

% We want to build a graph $G'$ with fewest possible vertices since fewer vertices means more parallelization. 

%\ishaq{TODO: @Ana: I am still confused about what prevents a reviewer from saying, "you argue correctness of the schedule from unrolled MPC, you don't argue it for the pre-unrolled CFG that you use". Please comment.} \ana{The correctness proof answer this question. The unrolled schedule is the \emph{concretization} of MPC source. We define (1) the obvious $\gamma$, (2) we define partial order over MPC source codes (this is our abstract domain $A$), and (3) a partial order over unrolled schedules. Then we prove two theorems that $A \le A' \Rightarrow \gamma(A) \subseteq \gamma(A')$ and that the MPC source $\le$ Vectorized MPC source.}

%\end{comment}

\subsection{Scheduling is NP-hard}
\label{sec:np}

\ishaq{TODO: need make amortized cost a function (like in the problem statement above), it is proving to be tricky.}

We consider two operations, call them $A$ and $M$. $A$ and $M$ are two abstract MPC instruction, but as an example, $A$ stands for the ADD MPC instruction and $M$ stands for the MUL instruction. Each instruction in the program is either an $A$-instruction or an $M$-instruction. In order to benefit from parallelization/amortization, we must schedule two or more $A$-instructions in the same parallel node (or two or more $M$-instructions in the same parallel node). We also assume that scheduling $A$-instructions in parallel with $M$-instruction does not benefit from amortization\footnote{this is not strictly true, but assuming it, e.g. as in \cite{Ishaq2019, Demmler2015ABYA, Mohassel2018}, helps with the exposition.}. It incurs the exact same cost as scheduling the $A$-instructions in a node $P_A$, scheduling the $M$-instructions in a node $P_M$, and having $P_A$ precede $P_M$ in the parallel schedule.

We use the following cost model:

\begin{enumerate}
    \item $A$ costs $\alpha$ units and $M$ costs $\beta$ unites.
    \item There is unlimited bandwidth i.e. a single $A$-instruction (or $M$-instruction) costs as much as $N$ amortized $A$-instructions (or $M$-instructions), concretely either $\alpha$ unites or $\beta$ units.
\end{enumerate}

Consider a loop body that consists of $n$ sequences: $S_1$, ... $S_n$ of $A$ and $M$ instructions. More precisely, the loop body is such that its instructions can be grouped into such sequences. $S_1$, ... $S_n$ can execute in parallel, however, all instructions within a sequence must execute sequentially. For example, consider the three sequences (the right arrow indicates a \emph{dependence}, meaning that the source node must execute before the target node): 

\begin{enumerate}
    \item $A \rightarrow M \rightarrow A$
    \item $A \rightarrow A \rightarrow A$
    \item $M \rightarrow A \rightarrow M$
\end{enumerate} 

A \emph{schedule} $P: P_1 \rightarrow P_2 \dots \rightarrow P_k$ is such that for each sequence $S_i$ in the set, if $S_i[j]$ precedes $S_i[j']$ in $S_i$ then $S_i[j]$ is scheduled in node $P_\ell$, $S_i[j']$ is scheduled in node $P_{\ell'}$, and $P_\ell$ precedes $P_{l\ell'}$ in $P$. 

The cost of a schedule $P$ is 

\begin{equation}
    \mathit{cost}(P) = \sum_{i=1}^k \mathit{cost}(P_i)
\end{equation}

where $\mathit{cost}(P_i) = \alpha$ if $P_i$ consists of $A$-instructions only, $\beta$ if $P_i$ consists of $M$-instructions only, and $\mathit{cost}(P_i) = \alpha + \beta$ if $P_i$ mixes $A$-instructions and $M$-instructions. 

The problem is to find a schedule $P$ with \emph{minimal cost}. For example, a schedule with minimal cost for the sequences above is \[ A(1), A(2) \rightarrow M(1), A(2), M(3) \rightarrow A(1), A(2), A(3) \rightarrow M(3) \] (The parentheses above indicate the sequence where the instruction comes from: (1), (2), or (3).) The cost of this schedule is $3\alpha + 2\beta$. 

The problem of finding a schedule $P$ with a minimal $cost(P)$ for a given loop body has been shown to be an NP-Hard problem, as it can be reduced to the problem of finding a \emph{shortest common supersequence}, a known NP-Hard problem\cite{Maier1978},\cite{Vazirani2010}. The shortest common supersequence problem is as follows: {\it given two or more sequences find the the shortest sequence that contains all of the original sequences.} This can be solved in $O(n^k)$ time, where $n$ is the cardinality of the longest sequence and $k$ is the number of sequences. For our problem $n$ is the maximum length of a node and $k$ is the number of total number of nodes.

To see the reduction, suppose $P$ is a schedule with minimal cost (computed by a black-box algorithm). We can derive a schedule $P'$ with the same cost as $P$, by mapping each mixed node $P_i \in P$ to two consecutive nodes in $P'$: an $A$-instruction node followed by an $M$-instruction node. Clearly, $P'$, which now is a sequence of $A$'s and $M$'s, is a supersequence of each sequence $S_i$, i.e., $P'$ is a common supersequence of $S_1 \dots S_n$. It is also a shortest common supersequence. To see this, let $X$ and $Y$ denote, respectively, the number of $A$ and $M$ nodes in $P'$. The cost of $P'$, and $P$, is $X \cdot \alpha + Y \cdot \beta$. Now suppose, there exists a shorter common supersequence, $P''$ that consists of $X'$ nodes of type $A$-instructions $Y'$ nodes of type $M$-instructions. Since $P''$ is shorter than $P'$, therefore $X' + Y' < X + Y$, and $X' \cdot \alpha + Y' \cdot \beta < X \cdot \alpha + Y \cdot \beta$ i.e. $\mathit{cost}(P'') < \mathit{cost}(P')$. But $\mathit{cost}(P') = \mathit{cost}(P)$ and $\mathit{cost}(P)$ is the optimal cost.  Therefore $\mathit{cost}(P'') < \mathit{cost}(P')$ is contradiction and no such $P''$ exists.


